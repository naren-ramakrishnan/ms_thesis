\documentclass{article}
\usepackage{cite}
\begin{document}
\title{Characterizing Elections Using Dynamic Query Expansion And Probabilistic Soft Logic}
\author{Aravindan Mahendiran}
%\email{aravind@vt.edu}
\maketitle
%\section{Abstract}
%\section{Introduction}
\section{Literature Review}
The last decade has seen a massive explosion of on-line data in all forms be it news articles,blogs
or social media like twitter or Facebook. 
Twitter alone which came into being only in 2006 has now grown so big
\footnote{As of May 7,2013 twitter has 555m active registered users with
135000 new users signing up everyday and approximately 1 billion tweets created every 5 days}
that it has come to be looked at as a treasure trove
of mine-able data.
With free API's to collect this data, the easy access to large volumes of data alone has piqued 
the interest of research scientists in this domain. 
Researchers have studied various real world time series and have shown strong correlations of 
on-line chatter to book sales, box office earnings and even stock prices.
~\cite{gruhl2005predictive,asur2010predicting,bollen2011twitter}. Not only did they 
find correlations but they also were able to "with reasonable accuracy" make forecasts about future
trends too. 
The more curious research is whether such on-line chatter be used to model the social, economic
and political landscape of a country. 
Bollen et al. ~\cite{bollen2011modeling} used a version of the well-established psychometric 
instrument- Profile of Mood States(POMS) to model the mood of twitter traffic and correlate it to
a number of social and economic events that occurred during the same time line. 
The results from this research instigated more researchers to study and quantify the political 
sentiment through social media and if possible even forecast election results.
% ask Naren if this paper is worth citing or not 
% ~\cite{diakopoulos2010characterizing] 
\paragraph{}
We divide the rest of this literature review into four parts. 
First, we look at a selection of methodologies aimed at predicting elections through
volume based approaches. 
Second, we look at more sophisticated approaches that aim to model the demographics of an election
to make informed decisions about such predictions. 
Then, we shall summarize a quite prevalent pessimistic view on such methodologies' 
and data sets' capability to predict elections. 
The last part of the this literature review is dedicated to reviewing methodologies like dynamic
query expansion to grow a vocabulary of terms about a paticular event. We also review a software
tool called Probabilistic Soft Logic.
\subsection{Volume based approaches}
In one of the most cited papers in this space, Tumasjan et al.  ~\cite{tumasjan2010predicting}
claim that \emph{ "The mere number of tweets reflect voter preferences and comes close to 
traditional polls.."} while predicting  the 2010 German federal election by counting
candidate mentions on twitter. 
They go on to strongly conclude that twitter can indeed be a valid indicator of political opinion. 
This was followed by ~\cite{o2010tweets,saez2011total,bermingham2011using,demartini2011analyzing}
all of which use a volume based approach combined with sentiment analysis. 
Both ~\cite{o2010tweets,bermingham2011using} fit a regression model to opinion polls with 
volume share and sentiment as independent variables and conclude that sentiment is a weak predictor
compared to share of volume. 
In general the experiments described in these publications count the occurrence of certain hand
filtered keywords in the "twittersphere" and classify such tweets as positive or negative using
a classifier trained on human annotated lexicons.
A little advanced sentiment classifiers also provide the likelihood that given sample of text
belongs to a empirically defined psychological and structural categories like anxiety, anger,
sadness etc. 
\subsection{Profile Modeling}
% modelling papers come here
The more sophisticated approach adapted in 
~\cite{livne2011party,conover2011predicting,diaz2012taking} is to model the candidates or
voters in the elections rather than compute the aggregated sentiment of the mass. 
Conover et al. build a SVM trained on manually labled tweets and hash-tags obtained
through co-occurrence discovery classify users into 'left' and 'right' aligned.
Through latent semantic analysis they claim to have identified the hidden structure 
in the data that is strongly associated with the political affiliations. 
Using this information and how political information diffuses in a network, they show 
an accuracy of 95\%  in predicting the political alignment of twitter users.
Livne et al. in ~\cite{livne2011party} analyse the twitter profiles of candidates from the 2010
mid term elections in the U.S. They identify topics specific to groups of candidates ,
split according to their known political orientations and use the features obtained as inputs
to a regression model to predict the elections.
In a similar technique Diaz-Aviles in ~\cite{diaz2012taking} model the candidates by building
a emotional vector for each candidate contesting in the elections by using the mentions
of that candidate and sentiments associated with each mention learnt using the NRC EmotionLexicon(EmoLex). 
They use these profiles learnt to predict the rise and fall of a candidate's popularity.
In another research, Mustafaraj et al. ~\cite{mustafaraj2011vocal} model the distribution 
of political content among users in twitter. They claim that there is a spectrum of users in twitter,
they dub them as "vocal minority" and "silent majority" and observe these two groups 
engage in different ways in social media. They show that, the vocal minority aim to
broaden the impact of tweets by re-tweeting and linking to outside links whereas 
the silent majority who tweet significantly lesser are more inclined to share their 
personal view points.
Though they do not make any predictions about elections, they make very valid observations
such as \emph{"Because of this differences between content generated by different groups ,
one should be aware of aggregating data and building models upon them, without verifying the
underlying model that has generated the data."}.

% ~\cite{sobkowicz2012opinion}....but try and understand what to summarize from the paper...paper in itself is very abstract.
% ~\cite{huang2012social,kimmig2012short} and write up about how this is the closely related to ur methodologies.
% ask Naren if modeling in PSL should be lit. review or part of methodology.

\subsection{Flaws in current state of the art and recommendations for the future}
Of late there has been a lot of studies showing how such models that predict elections using
social media feeds are flawed ~\cite{metaxas2011not,gayo2012wanted,gayo2011don,gayo2011limits}. 
These publications not only list the obvious issues in using twitter as a input to predict
elections but also detail recommendations on how to make such methodologies fool proof. 
Daniel Gayo-Avello surveys almost all the state of the art approaches in predicting elections in 
his paper ~\cite{gayo2012wanted} most of which is detailed above. 
According to him post-hoc analysis of elections in retrospect must not count as valid predictions
and also states that researchers do not report negative results leading to what is called 
the \emph{file drawer} effect. His major points of argument against such models are:
\begin{itemize}
\item
The models are tailor made to fit a particular election and that they need to be generic 
enough and must be able reproduce similar results when run on other elections.
In particular Metaxas et al in ~\cite{metaxas2011not} state that any method claiming predictive
power on the basis of Twitter data should be a clearly defined algorithm and 
should be "explainable" i.e., black box approaches should be avoided.
\item
There is no predefined notion of "vote" that has been used to predict the elections. 
Most of the models aim to predict elections merely based on volume of tweets related to
a candidate and sentimental analysis of such tweets.
\item
Biases in twitter are ignored. Twitter is not a representative sample of the electorate demographic
as that not every age gender or social group is represented. He also notes that since people
tweet on a voluntary basis the data produced is only by those who are politically active. 
Another important point of contention relating to bias is the credibility of tweets,
whether the tweets are rumors, campaign propaganda or contain misleading information just to
maliciously attack candidate's popularity on-line.
\item
Since in 2008 and 2010 , 91.6\% and 84\% of elections were won by the the incumbent candidate
respectively, Gayo-Avello argues that incumbency should be the baseline rather than just chance.
He also notes that most of the methodologies are only slightly better than chance.
\item
He goes on to argue sentiment analysis and classifiers, though a highly researched subject in NLP,
do not perform any better than random classifiers and that these sentiment analysis
techniques should be specifically built to handle the political corpus to detect humour and sarcasm
which in his opinion would play a major role.
\item
Lastly in ~\cite{gayo2011don} Gayo-Avello akin to ~\cite{mustafaraj2011vocal} states that
abstaining from tweeting about politics can play even more important role than the ones mentioning 
the candidates and hence researches should also model this lack of chatter about a particular
candidate or political party.
\end{itemize}
\subsection{Dynamic Query Expansion}
In ~\cite{liang2013dqe} the authors present an dynamic query expansion algorithm that given an
initial query of seed terms learns an expanded set of terms building from the seed terms but more
related to the ongoing events. They then use this newer vocabulary to predict events. We are however
interested only in the unsupervised algorithm to build a dictionary of keywords. The authors
provide mathematical proofs and experimental results to show that the dynamic query expansion 
algorithm developed converges. 
% Not sure if details of the DQE algorithm should go here or in the methodologies section.
\subsection{Probabilistic Soft Logic}
Huang et al. in ~\cite{huang2012social} use Probabilistic Soft Logic to characterize the political
scenario in Venezuela by performing analysis of user mentions and tracking keywords on Twitter.
Probabilistic Soft Logic is a framework for collective, probabilistic reasoning in relational
domains. It represents the domain of interest as logical atoms and uses first order logic rules
to capture the dependency structure of the domain, based on which it builds a joint 
probabilistic model over all items. PSL provides for inference methods for inferring the most
probable explanation or MPE inference and also computing the marginal distribution.
% Again not sure if a more detailed explanation of PSL should go into methodologies section.
\bibliography{references}
\bibliographystyle{plain}
\end{document}
